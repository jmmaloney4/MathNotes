\documentclass[11pt]{article}
\usepackage[utf8]{inputenc}	% Para caracteres en español
%\usepackage{multirow,booktabs}
\usepackage{geometry}
\usepackage{hyperref}

\geometry{margin=0.5in, headsep=0.25in}

\usepackage{base}

\renewcommand{\P}[1]{{\Phi}_{#1}}
\newcommand{\Pe}{\P{e}}
\newcommand{\PP}[2]{ \P{#1} \Parens{#2} }
\newcommand{\PeP}[1]{ \PP{e}{#1} }
\newcommand{\Cvgs}[1]{{#1 \converges}}
\newcommand{\Dvgs}[1]{{#1 \diverges}}


\title{Notes on \textit{Computability and Randomness} by André Nies \\
	\small{\href{https://ipfs.io/ipfs/bafykbzaceayfrh4ig622dtnwbws7osjrb2i6d5lum3ndtkxlqnwvrhcb7fqp6}
	{/ipfs/bafykbzaceayfrh4ig622dtnwbws7osjrb2i6d5lum3ndtkxlqnwvrhcb7fqp6}}}
\author{Jack Maloney}
\date{}

\begin{document}

\maketitle

\section{The Complexity of Sets}

\subsection{Computably Enumerable Sets}

\begin{align}
	W_e 
	= \text{dom} \Parens{\Pe} 
	= \Set{ i \in \N | \Cvgs{\PeP{i}} }
\end{align}

Then \( \Parens{W_e}_{e \in \N} \) is an effective listing of all c.e. sets.

\subsubsection{The Halting Set}

\begin{align}
	\emptyset' = \Set{ e | e \in W_e }
\end{align}

This set is many-one equivalent to the following set:

\begin{align}
	H = \Set{ \braket{e, n} | \Cvgs{\PeP{n} } }
\end{align}

\begin{proof}
	Given \( \braket{e,n} \in H \), by the parameter theorem we can effectively find \( \P{m} \simeq \PeP{n} \). Since \( \Cvgs{\PeP{n}} \), \( \Cvgs{ \P{m} } \). Thus \( W_m = \N \), so \( m \in W_m \), so \( m \in \emptyset' \). So \( H \, {\leq}_m \, \emptyset' \)\ by the computable function \( f(\braket{e, n}) = m \).
	
	Conversely, take \(e \in \emptyset' \). Then \( \Cvgs{\PeP{e}} \), so \( \braket{e,e} \in H \). So \( \emptyset' \, {\leq}_m \, H \) by the function \( f(e) = \braket{e,e} \).
\end{proof}

By Prop. 1.2.22, 

\begin{align}
	{\leq}_m \implies {\leq}_{tt} \implies {\leq}_{wtt} \implies {\leq}_{T}.  
\end{align}

So 
\begin{align}
	\emptyset' \equiv_m H \implies \emptyset' \equiv_T H.
\end{align}

\end{document}